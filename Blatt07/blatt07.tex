\newcommand{\authorinfo}{Julian Deinert, Frederik Wille, Carolin Konietzny}
\newcommand{\titleinfo}{ALA 07 (HA) zum 05.06.2014}

% PREAMBLE ===============================================================

\documentclass[a4paper,11pt,fleqn]{scrartcl}
\usepackage[german,ngerman]{babel}
\usepackage[utf8]{inputenc}
\usepackage[T1]{fontenc}
\usepackage{lmodern}
\usepackage{amssymb}
\usepackage{amsmath}
\usepackage{enumerate}
\usepackage{fancyhdr}
\usepackage{pgfplots}
\usepackage{multicol}
\usetikzlibrary{calc}
\usetikzlibrary{patterns}

\author{\authorinfo}
\title{\titleinfo}
\date{\today}

\newcommand{\bra}[1]{\left(#1\right)}
\newcommand{\limnn}[2]{\lim\limits_{n \rightarrow #1}\bra{#2}}
\newcommand{\limn}[1]{\lim\limits_{n \rightarrow \infty}\bra{#1}}
\newcommand{\limx}[1]{\lim\limits_{x \rightarrow \infty}\bra{#1}}
\newcommand{\limz}[1]{\lim\limits_{z \rightarrow \infty}\bra{#1}}
\newcommand{\rowi}[1]{\sum_{i=#1}^{\infty}}
\newcommand{\row}{\rowi{0}}
\newcommand{\step}[1]{\textbf{#1}}
\newcommand{\dX}[1]{\, \mathrm{d}#1}
\newcommand{\dx}[0]{\dX{x}}
\newcommand{\dt}[0]{\dX{t}}

\begin{document}
\maketitle
\begin{enumerate}
    
\item[\textbf{1.}]
\begin{enumerate}
    \item[(a)]
                
        Nur \(h(x)\) hat einen Wendepunkt:

        \( h(x) = \frac{1}{1+x^2} \)

        \( h'(x) = -\frac{2x}{(1+x^2)^2} \)

        \( h''(x) = \frac{-2(1+x^2)^2 + 8x^2(1+x^2)}{(1+x^2)^4} = \frac{6x^2-2}{(1+x^2)^4} \)

        Wendestelle bei \( 6x^2 - 2 = 0 \Leftrightarrow x = \pm \sqrt{\frac{1}{3}} \).

        Skizze der Graphen:

        \begin{tikzpicture}
            \begin{axis}[
                ymin=0,ymax=1.2,
                xmin=0,xmax=10.7,
                x=1cm, y=3cm,
                axis x line=middle,
                axis y line=middle,
                axis line style=->,
                xlabel={$x$},
                ylabel={$y$},
                ]
                \addplot[very thick,         no marks,   -] expression[domain=0:10,samples=100]{e^-x}; \addlegendentry{f(x)}
                \addplot[very thick, dotted, no marks, -] expression[domain=0:10,samples=100]{1/(1+x)}; \addlegendentry{g(x)}
                \addplot[very thick, dashed, no marks,  -] expression[domain=0:10,samples=100]{1/(1+x^2)}; \addlegendentry{h(x)}
            \end{axis}

            \node[fill=black, inner sep=1.4pt, circle, label=right:{\small Wendepunkt $h(x)$}] at (0.5773,2.25) {};
        \end{tikzpicture}

    \item[(b)]

        \[ \limz{\int_0^z e^{-x} \dx} = \limz{\left[ -e^{-x} \right]_0^z}  =
            \limz{e^0 - e^{-z}} = 1 \]

        \[ \limz{\int_0^z \frac{1}{1+x} \dx} = \limz{\left[ \ln (x+1) \right]_0^z}  =
            \limz{\ln(z+1)} - \ln(1) = \infty \]

        \[ \limz{\int_0^z \frac{1}{1+x^2} \dx} = \limz{\left[ \arctan x \right]_0^z}  =
            \limz{\arctan(z)} - \arctan(0) = \frac{\pi}{2} \]

    \item[(c)]
        Graph:

        \begin{tikzpicture}
            \begin{axis}[
                ymin=0,ymax=4,
                xmin=-1.3,xmax=1.3,
                x=2.5cm, y=1cm,
                axis x line=middle,
                axis y line=middle,
                axis line style=->,
                xlabel={$x$},
                ylabel={$y$},
                ]
                \addplot[very thick, no marks, black, -] expression[domain=-1:1,samples=100]{1/sqrt(1-x^2)};
            \end{axis}
        \end{tikzpicture}

        Die Funktion ist an der y-Achse spiegelsymmetrisch, deshalb gilt f"ur die Fl"ache:

        \[ A = 2 \cdot \int_0^1 \frac{1}{\sqrt{1-x^2}} \dx = 2 \left[ \arcsin(x) \right]_0^1 = 2 \bra{\arcsin(1)-\arcsin(0)} = 2(\frac{\pi}{2} - 0) = \pi \]

    \end{enumerate}

\item[\textbf{2.}]


\item[\textbf{3.}]
    \begin{enumerate}
        
    \item[a)]
        \[ 
        \lim_{i\rightarrow\infty} \left| \frac{a_{i+1}}{a_i}\right| =
        \lim_{i\rightarrow\infty} \left| \frac{(i+1)^2\cdot 2^{i+1}\cdot x^{i+1}}{i^2 2^i x^i}\right| =
        \lim_{i\rightarrow\infty} \left| \frac{(i+1)^2\cdot 2x}{i^2}\right| =
        2\left| x\right|\lim_{i\rightarrow\infty} \left| \frac{(i+1)^2}{i^2}\right| =
        2\left| x \right|
        \]

        Nach dem Quotientenkriterium konvergiert die Reihe f"ur $|x| < 0.5$, es gilt $R = 0.5$.

    \item[b)]
        \[
        \lim_{i\rightarrow\infty} \sqrt[i]{|a_i|} =
        \lim_{i\rightarrow\infty}  \sqrt[i]{|i^2 2^i x^i|} =
        \lim_{i\rightarrow\infty}  2|x|\sqrt[i]{|i^2|} =
        2|x|
        \]

        Nach dem Wurzelkriterium konvergiert die Reihe f"ur $|x| < 0.5$, es gilt $R = 0.5$.
    \end{enumerate}
    

\item[\textbf{4.}]


\item[\textbf{5.}]


\item[\textbf{6.}]

\end{enumerate}
\end{document}