\newcommand{\authorinfo}{Julian Deinert, Frederik Wille, Carolin Konietzny}
\newcommand{\titleinfo}{ALA 07 (HA) zum 05.06.2014}

% PREAMBLE ===============================================================

\documentclass[a4paper,11pt,fleqn]{scrartcl}
\usepackage[german,ngerman]{babel}
\usepackage[utf8]{inputenc}
\usepackage[T1]{fontenc}
\usepackage{lmodern}
\usepackage{amssymb}
\usepackage{amsmath}
\usepackage{enumerate}
\usepackage{fancyhdr}
\usepackage{pgfplots}
\usepackage{multicol}
\usetikzlibrary{calc}
\usetikzlibrary{patterns}

\author{\authorinfo}
\title{\titleinfo}
\date{\today}

\newcommand{\bra}[1]{\left(#1\right)}
\newcommand{\limnn}[2]{\lim\limits_{n \rightarrow #1}\bra{#2}}
\newcommand{\limn}[1]{\lim\limits_{n \rightarrow \infty}\bra{#1}}
\newcommand{\limx}[1]{\lim\limits_{x \rightarrow \infty}\bra{#1}}
\newcommand{\limz}[1]{\lim\limits_{z \rightarrow \infty}\bra{#1}}
\newcommand{\rowi}[1]{\sum_{i=#1}^{\infty}}
\newcommand{\row}{\rowi{0}}
\newcommand{\step}[1]{\textbf{#1}}
\newcommand{\dX}[1]{\, \mathrm{d}#1}
\newcommand{\dx}[0]{\dX{x}}
\newcommand{\dt}[0]{\dX{t}}

\begin{document}
\maketitle
\begin{enumerate}
    
\item[\textbf{1.}]


\item[\textbf{2.}]


\item[\textbf{3.}]
    

\item[\textbf{4.}]
	\begin{itemize}
		\item[a)]
			\begin{itemize}
				\item[i)]
					falsch, da laut Skript Polynome nicht so schnell wachsen wie Exponentialfunktionen
				\item[ii)]
					richtig, da bei Exponentialfunktionen mit gleichem Exponenten, die mit höherer Basis mindestens so schnell wächst wie die Funktion mit niedrigerer Basis
				\item[iii)]
					richtig, da bei Exponentialfunktionen die Basis für sehr hohe Exponenten kaum einen Unterschied machen
				\item[iv)]
					richtig, denn der Exponent $\sqrt{n}$ ist bedeutend kleiner als n
				\item[v)]
					falsch, durch die Wurzel wächst $\log _2 \left(\sqrt{n}\right)$ deutlich langsamer 
				\item[vi)]
					falsch, da ebenfalls durch die Wurzel die rechte Funktion langsamer wächst
				
			\end{itemize}
		\item[b)]
			$f(x)=n$ \\
			$g(x)=sin\left(x\right) $
%			Da g(x) nicht "asymptotisch nicht-negativ" ist
	\end{itemize}

\end{enumerate}
\end{document}
