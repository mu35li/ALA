\newcommand{\authorinfo}{Julian Deinert, Frederik Wille, Carolin Konietzny}
\newcommand{\titleinfo}{ALA 07 (HA) zum 05.06.2014}

% PREAMBLE ===============================================================

\documentclass[a4paper,11pt,fleqn]{scrartcl}
\usepackage[german,ngerman]{babel}
\usepackage[utf8]{inputenc}
\usepackage[T1]{fontenc}
\usepackage{lmodern}
\usepackage{amssymb}
\usepackage{amsmath}
\usepackage{enumerate}
\usepackage{fancyhdr}
\usepackage{pgfplots}
\usepackage{multicol}
\usetikzlibrary{calc}
\usetikzlibrary{patterns}

\author{\authorinfo}
\title{\titleinfo}
\date{\today}

\newcommand{\bra}[1]{\left(#1\right)}
\newcommand{\limnn}[2]{\lim\limits_{n \rightarrow #1}\bra{#2}}
\newcommand{\limn}[1]{\lim\limits_{n \rightarrow \infty}\bra{#1}}
\newcommand{\limx}[1]{\lim\limits_{x \rightarrow \infty}\bra{#1}}
\newcommand{\limz}[1]{\lim\limits_{z \rightarrow \infty}\bra{#1}}
\newcommand{\rowi}[1]{\sum_{i=#1}^{\infty}}
\newcommand{\row}{\rowi{0}}
\newcommand{\step}[1]{\textbf{#1}}
\newcommand{\dX}[1]{\, \mathrm{d}#1}
\newcommand{\dx}[0]{\dX{x}}
\newcommand{\dt}[0]{\dX{t}}

\begin{document}
\maketitle
\begin{enumerate}
    
\item[\textbf{1.}]

	\begin{enumerate}
		\item[{a)}]
			 $f(x)=cos(x)$ 
			 $x_0=0$ \\
			 $T_7(x)=1-\frac{x^2}{2}+\frac{x^4}{24}-\frac{x^6}{720}$ \\
			 $T_8(x)=T_7(x)+\frac{cos(0)}{5040}*x^8$\\
			 $T_8(x)=T_7(x)+\frac{x^8}{5040}=1-\frac{x^2}{2}+\frac{x^4}{24}-\frac{x^6}{720}+\frac{x^8}{5040}$\\
			 $T_9(x)=T_8(x)$\\
			 $T_{10}(x)=T_9(x)+\frac{-cos(0)}{362800}*x^10=1-\frac{x^2}{2}+\frac{x^4}{24}-\frac{x^6}{720}+\frac{x^8}{5040}-\frac{x^10}{362800}$\\
			 $T_{11}(x)=T_{10}(x)$\\
			 $T_{12}(x)=T_{11}(x)+\frac{cos(o)}{479001600}=1-\frac{x^2}{2}+\frac{x^4}{24}-\frac{x^6}{720}+\frac{x^8}{5040}-\frac{x^{10}}{362800}+\frac{1^{12}}{479001600}$\\
			 $T_{13}(x)=T_{12}(x)$\\ \\
			 $T_9(1)=1-\frac{1^2}{2}+\frac{1^4}{24}-\frac{1^6}{720}+\frac{1^8}{5040}
			 =1-\frac{1}{2}+\frac{1}{24}-\frac{1}{720}+\frac{1}{5040}=\frac{227}{420} 
			 \approx 0,54\overline{047619}$\\ \\
			 $T_{11}(1)=1-\frac{1^2}{2}+\frac{1^4}{24}-\frac{1^6}{720}+\frac{1^8}{5040}-\frac{1^{10}}{362800}
			 =1-\frac{1}{2}+\frac{1}{24}-\frac{1}{720}+\frac{1}{5040}-\frac{1}{362800}$ \\
			 $T_{11}(1) \approx 0,5404734341$ \\ \\
			 $T_{13}(1)=T_{11}(1)+\frac{1^{13}}{479001600} \approx 0,5404734362$\\
		\item[b)]
			$f(x)=\sqrt{1+x}$\\ \\
			$T_0(x)=\frac{\sqrt{1+0}}{1}*1=\sqrt{1}=1$\\
			$T_1(x)=T_0(x)+\frac{\frac{1}{2}*\sqrt{1+0}}{1}*x=1+\frac{1}{2}*x$\\
			$T_2(x)=T_1(x)+\frac{x^2}{4}=1+\frac{1}{2}*x+\frac{x^2}{4}$\\
			$T_3(x)=T_2(x)+\frac{x^3}{8}=1+\frac{1}{2}*x+\frac{x^2}{4}+\frac{x^3}{8}$\\
			$T_4(x)=T_3(x)+\frac{x^4}{16}=1+\frac{1}{2}*x+\frac{x^2}{4}+\frac{x^3}{8}+\frac{x^4}{16}$\\ \\ \\
			$g(x)=\frac{1}{\sqrt[3]{1+x}}$\\ \\
			$T_0(x)=\frac{1}{\sqrt[3]{1+0}*1}*x^0=\frac{1}{1}=1$\\
			$T_1(x)=T_1(x)+\frac{1}{3* \sqrt[3]{1}}*x=1+\frac{x}{3}$\\
			$T_2(x)=T_2(x)+\frac{1}{9* \sqrt[3]{1}}=1+\frac{x}{3}+\frac{x^2}{9}$\\
			$T_3(x)=1+\frac{x}{3}+\frac{x^2}{9}+\frac{x^3}{27}$\\
			$T_4(x)=1+\frac{x}{3}+\frac{x^2}{9}+\frac{x^3}{27}+\frac{x^4}{81}$\\
		\item[c)]
			$f(x)=e^x*\sin(x)$\\
			$T_5(x)={e^0*\sin(0)}+{e^0*(\cos(0)+\sin(0))}x+\frac{2e^0\cos(0)}{2}x^2+\frac{2e^0\left(\cos(0)-\sin(0)\right)}{6}x^3+\frac{-4e^0\sin(0)}{24}x^4+\frac{-4e^0\left(\sin(0)+\cos(0)\right)}{120}x^5$\\
			$T_5(x)=x+x^2+\frac{2}{6}x^3+\frac{-4}{120}x^5$\\
	\end{enumerate}

% Aufgabe 2
\item[\textbf{2.}]
    \begin{enumerate}
        \item[(i)]
            \begin{align*}
                &\underset{x \to 1}{\lim} \left( \frac{x^3-3x^2+x+2}{x^2-5x+6} \right) = \frac{1}{2}
            \end{align*}
        \item[(ii)]
            \begin{align*}
                &\underset{x \to 2}{\lim} \left( \frac{x^3-3x^2+x+2}{x^2-5x+6} \right)\\
                \overset{*}{=}&\underset{x \to 2}{\lim} \left( \frac{3x^2-6x+1}{2x-5} \right)\\
                =& -1
            \end{align*}
            * Anwendung der Regel von Los Hospitalos (aka l'Hospital)..
        \item[(iii)]
            \begin{align*}
                &\underset{x \to 0}{\lim} (1+3x)^{\frac{1}{2x}}\\
                =&\underset{x \to 0}{\lim} \left( e^{\frac{1}{2x} \cdot \ln(1+3x)} \right)\\
                =&e^{\underset{x \to 0}{\lim} \left( \frac{1}{2x} \cdot \ln(1+3x) \right)}
            \end{align*}
            Berechnung des Exponenten:
            \begin{align*}
                &\underset{x \to 0}{\lim} \left( \frac{\ln(1+3x)}{2x} \right)\\
                \overset{*}{=}&\underset{x \to 0}{\lim} \left( \frac{\frac{3}{3x+1}}{2} \right)\\
                =& \frac{3}{2}\\
            \end{align*}
                Es folgt: $e^{\frac{3}{2}}$\\\\
            * Hier wieder l'Hospital..
        \item[(iv)]
            \begin{align*}
                &\underset{x \to 0}{\lim} \left( \frac{1}{e^x -1}-\frac{1}{\sin(x)} \right)\\
                =&\underset{x \to 0}{\lim} \left( \frac{\sin(x) - e^x + 1}{e^x \sin(x) - \sin(x)} \right)\\
                \overset{*}{=}&\underset{x \to 0}{\lim} \left( \frac{\cos(x) - e^x}{e^x \sin(x) +(e^x -1) \cos(x)} \right)\\
                \overset{*}{=}&\underset{x \to 0}{\lim} \left( \frac{-e^x - \sin(x)}{\sin(x) +2e^x \cos(x)} \right)\\
                =& -\frac{1}{2}
            \end{align*}
            * de l'Hospital again.
    \end{enumerate}

% Aufgabe 3
\item[\textbf{3.}]
    \begin{enumerate}
        \item[a)]
            Die Steigung von $t$ ist die Steigung von $f$ an der Stelle $(2,9)$.
            \begin{align*}
                f(x)=x^3-x^2+3x-1\\
                f'(x)=3x^2-2x+3\\
                f'(2)=11
            \end{align*}
            Die Steigung von $t$ ist 11. Daraus ergibt sich $t(2)=9$, oder ebenso $t(3)=20$. Also:
            \begin{align*}
                t(x)=ax+b\\
                9=2a+b\\
                20=3a+b\\
                b=9-2a\\
                20=3a+9-2a\\
                20=a+9\\
                a=11\\
                b=-13\\
                t(x)=11x-13\\
                t(x)=0\\
                \Leftrightarrow x=\frac{13}{11}
            \end{align*}
            $t$ hat den Schnittpunkt mit der $x$-Achse an der Stelle $\frac{13}{11}$
        \item[c)]
            Zuerst berechnet man die ersten drei Ableitungen von $\sqrt[5]{x+1}$ und die Funktionswerte für $x = 0$,
            danach wird dies in die Taylorpolynomformel eingesetzt.
            \begin{align*}
            &f(x) = \sqrt[5]{x+1} && f(0) = 1 \\
            &f'(x) = \frac{1}{5}(x+1)^{-\frac{4}{5}} && f'(0) = \frac{1}{5} \\
            &f''(x) = -\frac{4}{25}(x+1)^{-\frac{9}{5}} && f''(0) = -\frac{4}{25} \\
            &f'''(x) = \frac{36}{125}(x+1)^{-\frac{14}{5}} && f'''(0) = \frac{36}{125} \\
            \end{align*}
            Einsetzen in $$\sum_{k=0}^{n} \frac{f^{(k)}(0)}{k!}x^k:$$ 
            \begin{align*}
            &T_0(x) = 1 \\
            &T_1(x) = 1 + \frac{1}{5}x \\
            &T_2(x) = 1 + \frac{1}{5}x -\frac{2}{25}x^2 \\
            &T_3(x) = 1 + \frac{1}{5}x -\frac{2}{25}x^2 + \frac{6}{125}x^3 \\
            \end{align*}
        \item[d)]
            Die Funktion $h$ ist genau dann differenzierbar, wenn der Grenzwert des Differenzenquotienten existiert (mit $x_0=0$):
            \begin{align*}
                &\underset{x \to 0}{\lim} \frac{h(x)-h(0)}{x}\\
                =\ &\underset{x \to 0}{\lim} \frac{x \cdot \cos(\frac{1}{x})}{x}\\
                \overset{*}{=}\ &\underset{x \to 0}{\lim} \frac{\sin(\frac{1}{x})}{x} + \cos{\frac{1}{x}}
            \end{align*}
            *de l'Hospital\\
            Die Funktion an der Stelle 0 nicht differenzierbar, da der Grenzwert zwischen $-\infty$ und $\infty$ schwankt.
    \end{enumerate}

    

\item[\textbf{4.}]


\item[\textbf{5.}]
%	$\int_a^b \! f(x) \, \mathrm{d}x$\\
	\begin{eqnarray*}
		\int_0^1 \! \sin(x) \, \mathrm{d}x \approx \frac{1-0}{2n}*\sum\limits_{i=1}^n(f(x_{i-1}+f(x_i))\\
	\end{eqnarray*}
	\begin{itemize}
		\item[n=4]
			\begin{align*}
				\int_0^1 \! \sin(x) \, \mathrm{d}x &\approx \frac{1-0}{8}*\left(f(x_0)+2f(x_1)+2f(x_2)+2f(x_3)+f(x_4)\right)\\
				&= \frac{1}{8}*\left(\sin\left(0\right)+2\sin\left(\frac{1}{4}\right)+2\sin\left(\frac{1}{2}\right)+2\sin\left(\frac{3}{4}\right)+\sin(1)\right) \\
				&= \frac{1}{8}*3.65841 \\
				&= 0.4573
			\end{align*}
		\item[n=5]
			\begin{align*}
				\int_0^1 \! \sin(x) \, \mathrm{d}x &\approx \frac{1-0}{10}*\left(f(x_0)+2f(x_1)+2f(x_2)+2f(x_3)+2f(x_4)+f(x_5)\right)\\
				&= \frac{1}{10}*\left(\sin\left(0\right)+2\sin\left(\frac{1}{5}\right)+2\sin\left(\frac{2}{5}\right)+2\sin\left(\frac{3}{5}\right)+2\sin\left(\frac{4}{5}\right)+\sin(1)\right) \\
				&= 0.4581
			\end{align*}
		\item[n=10]
			\begin{align*}
				\int_0^1 \! \sin(x) \, \mathrm{d}x &\approx \frac{1-0}{20}*(f(x_0)+2f(x_1)+2f(x_2)+2f(x_3)+2f(x_4)+2f(x_5)\\ &+2f(x_6)+2f(x_7)
+2f(x_8)+2f(x_9)+f(x_{10}))\\
				&\approx \frac{1}{10}*(sin(0)+2\sin\left(\frac{1}{10}\right)+2\sin\left(\frac{2}{10}\right)+2\sin\left(\frac{3}{10}\right)+2\sin\left(\frac{4}{10}\right)\\ &+2\sin\left(\frac{5}{10}\right)+2\sin\left(\frac{6}{10}\right)+2\sin\left(\frac{7}{10}\right)+2\sin\left(\frac{8}{10}\right)+2\sin\left(\frac{9}{10}\right)+\sin\left(1\right)) \\
				&\approx \frac{1}{20} * 9,1862 \\
				&\approx 0,45931
			\end{align*}
	\end{itemize}
            


\item[\textbf{6.}]

\end{enumerate}
\end{document}
