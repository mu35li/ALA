\newcommand{\authorinfo}{Julian Deinert, Frederik Wille, Carolin Konietzny}
\newcommand{\titleinfo}{ALA 10 (HA) zum 03.07.2014}

% PREAMBLE ===============================================================

\documentclass[a4paper,11pt,fleqn]{scrartcl}
\usepackage[german,ngerman]{babel}
\usepackage[utf8]{inputenc}
\usepackage[T1]{fontenc}
\usepackage{lmodern}
\usepackage{amssymb}
\usepackage{amsmath}
\usepackage{enumerate}
\usepackage{fancyhdr}
\usepackage{pgfplots}
\usepackage{multicol}
\usetikzlibrary{calc}
\usetikzlibrary{patterns}

\author{\authorinfo}
\title{\titleinfo}
\date{\today}

\newcommand{\bra}[1]{\left(#1\right)}
\newcommand{\limnn}[2]{\lim\limits_{n \rightarrow #1}\bra{#2}}
\newcommand{\limn}[1]{\lim\limits_{n \rightarrow \infty}\bra{#1}}
\newcommand{\limx}[1]{\lim\limits_{x \rightarrow \infty}\bra{#1}}
\newcommand{\limz}[1]{\lim\limits_{z \rightarrow \infty}\bra{#1}}
\newcommand{\rowi}[1]{\sum_{i=#1}^{\infty}}
\newcommand{\row}{\rowi{0}}
\newcommand{\step}[1]{\textbf{#1}}
\newcommand{\dX}[1]{\, \mathrm{d}#1}
\newcommand{\dx}[0]{\dX{x}}
\newcommand{\dt}[0]{\dX{t}}

\begin{document}
\maketitle
\begin{enumerate}
    
\item[\textbf{1.}]
    \begin{enumerate}
        \item[a)]
            \begin{align*}
                f_x &= 4x -2\\
                f_y &= 2y -2z -6\\
                f_z &= 8z -2y\\
            \end{align*}
            \begin{align*}
                f_{xx} &= 4  & f_{xy}  &= 0 & f_{xz} &= 0\\
                f_{yx} &= 0  & f_{yy}  &= 2 & f_{yz} &= -2\\
                f_{zx} &= 0  & f_{zy}  &= -2 & f_{zz} &= 8\\
            \end{align*}
                grad(0,0,0)
            \begin{align*}
                4x -2 &= 0   \\
                2y -2z-6 &= 0\\
                8z - 2y &= 0 \\
            \end{align*}
            Das Gleichungssystem hat die L"osung $S(\frac{1}{2}, 4, 1)$.

            \[ H = \begin{pmatrix} 4 & 0 & 0\\ 0 & 2 & -2\\ 0 & -2 & 8 \end{pmatrix}\]
            \begin{align*}
                \delta_1 &= 4 \\
                \delta_2 &= 8\\
                \delta_3 &= 34\\
            \end{align*}

            Die Matrix ist positiv definit, also ist an der Stelle $S(\frac{1}{2}, 4, 1)$ ein lokales Minimum. 
        
        \item[b)]
            \[\text{gradT}(x,y,z) = (4x -2 , 2y -2z -6, 8z -2y)\]

            An $P(1,1,1)$ gilt:

            \[\text{gradT}(1,1,1) = (2, -6, 6)\]

            Die Temperatur steigt in die Richtung des Vektors \( (2, -6, 6) \) am schnellsten an.
            Die Gr"osse des Anstiegs betr"agt 

            \[\left|\sqrt{2^2 + (-6)^2 + 6^2}\right| = 8,717797887\]

            Die Richtung des steilsten Abstiegs ist 

            \[-gradT(x,y,z) = (-(4x-2), -(2y-2z-6), -(8z-2y)) = (-2,6,-6)\]

    \end{enumerate}

\item[\textbf{2.}]
    
\item[\textbf{3.}]
    
\item[\textbf{4.}]
    
\end{enumerate}
\end{document}
