\newcommand{\authorinfo}{Julian Deinert, Frederik Wille, Carolin Konietzny}
\newcommand{\titleinfo}{ALA 07 (HA) zum 3.7.2014}

% PREAMBLE ===============================================================

\documentclass[a4paper,11pt,fleqn]{scrartcl}
\usepackage[german,ngerman]{babel}
\usepackage[utf8]{inputenc}
\usepackage[T1]{fontenc}
\usepackage{lmodern}
\usepackage{amssymb}
\usepackage{amsmath}
\usepackage{enumerate}
\usepackage{fancyhdr}
\usepackage{pgfplots}
\usepackage{multicol}
\usetikzlibrary{calc}
\usetikzlibrary{patterns}

\author{\authorinfo}
\title{\titleinfo}
\date{\today}

\newcommand{\bra}[1]{\left(#1\right)}
\newcommand{\limnn}[2]{\lim\limits_{n \rightarrow #1}\bra{#2}}
\newcommand{\limn}[1]{\lim\limits_{n \rightarrow \infty}\bra{#1}}
\newcommand{\limx}[1]{\lim\limits_{x \rightarrow \infty}\bra{#1}}
\newcommand{\limz}[1]{\lim\limits_{z \rightarrow \infty}\bra{#1}}
\newcommand{\rowi}[1]{\sum_{i=#1}^{\infty}}
\newcommand{\row}{\rowi{0}}
\newcommand{\step}[1]{\textbf{#1}}
\newcommand{\dX}[1]{\, \mathrm{d}#1}
\newcommand{\dx}[0]{\dX{x}}
\newcommand{\dt}[0]{\dX{t}}

\begin{document}
\maketitle
\begin{enumerate}
    
\item[\textbf{1.}]


\item[\textbf{2.}]


\item[\textbf{3.}]
    \begin{align*}
    	f(x,y)&=-0,2x^2-0,2xy-0,1y^2+48x+47y-500 \\
    	x+y&=200 \\
    	0&=x+y-200
    \end{align*}
    \begin{itemize}
    	\item[a)]
			$L(x,y,\lambda )=-0,2x^2-0,2xy-0,1y^2+48x+47y-500+ \lambda *\left( x+y-200 \right) $\\
			\begin{align*}
				\frac{\delta L}{\delta x}(x,y,\lambda)=-0,4x-0,2y+48+\lambda&=0\\
				\frac{\delta L}{\delta x}(x,y,\lambda)=-0,2x-0,2y+47+\lambda&=0\\
				x+y-200&=0
			\end{align*}
			\begin{align*}
				-0,2x+1&=0\\
				x&=5\\
				y&=200-x=195
			\end{align*}
			kritische Stelle bei (5,195)
		\item[b)]
			$\bar{H}=\left(\begin{matrix}
				0 & 1 & 1 \\
				1 & -0,4 & -0,2 \\
				1 & -0,2 & -0,2 
			\end{matrix}\right)$\\
			$|\bar{H}|=\frac{1}{5} \Rightarrow lokales Maximum $
    \end{itemize}

\item[\textbf{4.}]
	\begin{align*}
		f(x,y)&=-0,2x^2-0,2xy-0,1y^2+48x+47y-500 \\
		f(x)&=-0,2x^2-0,2x\left(200-x\right)-0,1\left(200-x\right)^2+48x+47\left(200-x\right)-500 \\
		&=-0,1x^2+x+4900 \\
		f'(x)&=-0,2x+1 \\
		0&=-0,2x+1\\
		1&=0,2x\\
		x&=5\\
		y&=200-x=195\\
	\end{align*}
\end{enumerate}
\end{document}
