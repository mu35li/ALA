\newcommand{\authorinfo}{Julian Deinert, Frederik Wille, Carolin Konietzny}
\newcommand{\titleinfo}{ALA 09 (HA) zum 26.06.2014}

% PREAMBLE ===============================================================

\documentclass[a4paper,11pt,fleqn]{scrartcl}
\usepackage[german,ngerman]{babel}
\usepackage[utf8]{inputenc}
\usepackage[T1]{fontenc}
\usepackage{lmodern}
\usepackage{amssymb}
\usepackage{amsmath}
\usepackage{enumerate}
\usepackage{fancyhdr}
\usepackage{pgfplots}
\usepackage{multicol}
\usetikzlibrary{calc}
\usetikzlibrary{patterns}

\author{\authorinfo}
\title{\titleinfo}
\date{\today}

\newcommand{\bra}[1]{\left(#1\right)}
\newcommand{\limnn}[2]{\lim\limits_{n \rightarrow #1}\bra{#2}}
\newcommand{\limn}[1]{\lim\limits_{n \rightarrow \infty}\bra{#1}}
\newcommand{\limx}[1]{\lim\limits_{x \rightarrow \infty}\bra{#1}}
\newcommand{\limz}[1]{\lim\limits_{z \rightarrow \infty}\bra{#1}}
\newcommand{\rowi}[1]{\sum_{i=#1}^{\infty}}
\newcommand{\row}{\rowi{0}}
\newcommand{\step}[1]{\textbf{#1}}
\newcommand{\dX}[1]{\, \mathrm{d}#1}
\newcommand{\dx}[0]{\dX{x}}
\newcommand{\dt}[0]{\dX{t}}

\begin{document}
\maketitle
\begin{enumerate}
    
\item[\textbf{1.}]
	\begin{itemize}
	\item[(i)]
		\begin{align*}
			f(x,y)&=2x^2y^2-3xy+4x+2\\
			\frac{\delta f}{\delta x}(x,y)&=4xy^2-3y+4\\
			\frac{\delta f}{\delta y}(x,y)&=4x^2y-3x\\
			&=4x^2y+x\\
		\end{align*}
	\item[(ii)]
		\begin{align*}
			f(x,y)&=\cos(x^2y)*e^{xy}\\
			\frac{\delta f}{\delta x}(x,y)&=-sin(x^2y)*e^{xy}*2xy+\cos(x^2y)*e^{xy}\\
			\frac{\delta f}{\delta y}(x,y)&=-sin(x^2y)*e^{xy}*x^2+\cos(x^2y)*e^{xy}\\
		\end{align*}
	\item[(iii)]
		\begin{align*}
			f(x,y)&=\frac{\sin(x)+\cos(y)}{x^2+y^2}\\
			\frac{\delta f}{\delta x}(x,y)&=\frac{ \left( \cos \left( x \right)+cos\left(y\right)\right)*\left(x^2+y^2\right)-\left(\sin(x)+\cos(y)\right)*2x }{\left(x^2+y^2\right)^2}\\
			\frac{\delta f}{\delta y}(x,y)&=\frac{\left(\sin (x)-\sin (y)\right)*\left(x^2+y^2\right)-\left(\sin(x)+\cos(y)\right)*2y}{\left(x^2+y^2\right)^2}
		\end{align*}
	\item[(iv)]
		\begin{align*}
			f(x,y)&=\sqrt{1-x^2-y^2}\\
			&=\left(1-x^2-y^2\right)^\frac{1}{2}\\
			\frac{\delta f}{\delta x}(x,y)&=\frac{1}{2}*\left(1-x^2-y^2\right)^{-\frac{1}{2}}*2x\\
			\frac{\delta f}{\delta y}(x,y)&=\frac{1}{2}*\left(1-x^2-y^2\right)^{-\frac{1}{2}}*2y
		\end{align*}
	\end{itemize}

\item[\textbf{2.}]
	\begin{align*}
		f(x,y)&=x^2y^3+ye^{x^2y}\\
		\frac{\delta f}{\delta x}(x,y)&=2xy^3+y^2*e^{x^2y}*2x\\
		\frac{\delta f}{\delta y}(x,y)&=3x^2y^2+x^2ye^{x^2y}\\
		\frac{\delta^2 f}{\delta x^2}(x,y)&=2y^3+y^2*e^{x^2y}*4x^2\\
		\frac{\delta^2 f}{\delta y^2}(x,y)&=6x^2y+x^2y^2e^{x^2y}\\
		\frac{\delta^2 f}{\delta x \delta y}(x,y)&=\left( \frac{\delta f}{\delta x}(x,y) \right)'=6xy^2+2x*\left( 2y*e^{x^2y} + ye^{x^2y}*x^2 \right)\\
		&=6xy^2+4xye^{x^2y}+2x^3e^{x^2y}\\
		\frac{\delta^2 f}{\delta y \delta x}(x,y)&=\left( \frac{\delta f}{\delta y}(x,y) \right)'=6xy^2+y*\left( 4xe^{x^2y}+2x^3ye^{x^2y} \right)\\
		&=6xy^2+4xye^{x^2y}+2x^3e^{x^2y}
	\end{align*}

\item[\textbf{3.}]
	\item[(i)]
	\begin{align*}
		f(x,y) &= -x^2-y^2+xy+x+6\\
		f_{x} &= -2x + y + 1\\
		f_{y}&= -2y +x\\
		f_{xx} &= -2\\
		f_{xy} &= 1\\
		f_{yx} &= 1\\
		f_{yy} &= 2\\\\
		\text{I }&-2x +y+1=0\\
		\text{II }&-2y+x=0\\
		\text{2} \times \text{I + II} &-4x+2y+2-2y+x=0\\
		&-3x+2=0\\
		&-3x=-2 \rightarrow x =\frac{2}{3} \text{in II einsetzen}\\
		&-2y+ \frac{2}{3} = 0\\
		&-2y = -\frac{2}{3} \rightarrow y = \frac{1}{3}\\
		S\left(\frac{2}{3},\frac{1}{3}\right)\\
	\end{align*}
	$\left(\begin{matrix}
 		 -1 & 3 \\
 		 2 & -4
	 \end{matrix}\right)$\\\\
	 $ \delta_{1}=-2\\ \delta_{2}=2$\\
	 Somit liegt bei S ein Maximum vor.\\
	 
	 \item[(ii)]
	 \begin{align*}
	 	f(x,y)&=3x^2+2y^2-xy-4x+y+1\\
	 	f_{x}&=6x-y-4\\
	 	f_{y}&=4y-x+1\\
	 	f_{xx} &= 6\\
		f_{xy} &= -1\\
		f_{yx} &= -1\\
		f_{yy} &= 4\\\\
		\text{I }&6x-y-4=0\\
		\text{II }&4y-x+1=0\\
		\text{4}\times\text{I + II}&24x-4y-16+4y-x+1=0\\
		&23x-15=0\\
		&23x=15\\
		&x=\frac{15}{23}\text{einsetzen in II}\\
		&4y-\frac{15}{23}+1=0\\
		&4y=-1+\frac{15}{23}\\
		&y=-\frac{2}{23}\\
		S\left(\frac{15}{23},-\frac{2}{23}\right)\\
	 \end{align*}
	 $\left(\begin{matrix}
 		 6 & -1 \\
 		 -1 & 4
	 \end{matrix}\right)$\\\\
	 $ \delta_{1}=6\\ \delta_{2}=26$\\
	 Somit liegt bei S ein Minimum vor.\\
	 
	 \item[(iii)]
	 \begin{align*}
	 	f(x,y)&=3x^2+y^2+4xy-x+y+2\\
	 	f_{x}&=6x+4y-1\\
	 	f_{y}&=2y+4x+1\\
	 	f_{xx} &= 6\\
		f_{xy} &= 4\\
		f_{yx} &=4\\
		f_{yy} &= 2\\\\
		\text{I }&6x+4y-1=0\\
		\text{II }&2y+4x+1=0\\
		\text{I - 2}\times\text{II}&(6x+4y-1)-(4y-8x+2)=0\\
		&6x+4y-1-4y+8x-2=0\\
		&14x-3=0\\
		&14x=3 \rightarrow x=\frac{3}{14}\\
		&2y+4*\frac{3}{14}+1 =0\\
		&2y= -\frac{13}{7} \rightarrow y =-\frac{13}{14}\\
		S\left(\frac{3}{14},-\frac{13}{14}\right)\\
	 \end{align*}
	 $\left(\begin{matrix}
 		 6 & 4 \\
 		 4 & 2
	 \end{matrix}\right)$\\\\
	 $ \delta_{1}=6\\ \delta_{2}=12-16 = -4$\\
	Daher indefinit.\\
	
	\item[(iv)]
	 \begin{align*}
	 	f(x,y)&=x^3+4y^3-9x-48y+7\\
	 	f_{x}&=3x^2-9\\
	 	f_{y}&=12y-48\\
	 	f_{xx} &= 6x\\
		f_{xy} &= 0\\
		f_{yx} &=0\\
		f_{yy} &= 24y x^2-3\\\\
		\text{I }&3x^2-9=0  | :3\\
		\text{II }&12y^2-48=0 | :12\\
		&x_{1/2} = 0 \pm \sqrt{3}\\
		&y_{1/2}=0 \pm \sqrt{4}\\
		&y_{1} = 2 y_{2}=-2\\
		&x_{1}=\sqrt{3} x_{2}-\sqrt{3}\\
		S_{1}\left(\sqrt{3},2\right) S_{2}\left(-\sqrt{3},-2\right)\\
	 \end{align*}
	 $H_{i}=\left(\begin{matrix}
 		 6_{xi} & 0 \\
 		 0 & 24_{yi}
	 \end{matrix}\right)$\\\\
	 $H_{1}=\left(\begin{matrix}
 		 6*\sqrt{3} & 0 \\
 		 0 & 24*2
	 \end{matrix}\right)$\\
	 $ \delta_{1}=6*\sqrt{3}\\ \delta_{2}= 438,8306326$\\
	 $H_{2}=\left(\begin{matrix}
 		 6*(-\sqrt{3}) & 0 \\
 		 0 & 24*(-2)
	 \end{matrix}\right)$\\
	 $ \delta_{1}=-6*\sqrt{3}\\ \delta_{2}=438,8306326$\\
	Es liegt ein Maximum vor.
	.\\

\item[\textbf{4.}]

\end{enumerate}
\end{document}
